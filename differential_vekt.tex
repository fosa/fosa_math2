% coding:utf-8

%----------------------------------------
%FOSAPHY, a LaTeX-Code for a summary of mathematics
%Copyright (C) 2013, Mario Felder, Michael Fallegger

%This program is free software; you can redistribute it and/or
%modify it under the terms of the GNU General Public License
%as published by the Free Software Foundation; either version 2
%of the License, or (at your option) any later version.

%This program is distributed in the hope that it will be useful,
%but WITHOUT ANY WARRANTY; without even the implied warranty of
%MERCHANTABILITY or FITNESS FOR A PARTICULAR PURPOSE.  See the
%GNU General Public License for more details.
%----------------------------------------

\chapter{Differential Funktionen}
\section{Tangentialebene, Linearisierung, totales Differential}
Die Gleichung der \textbf{Tangentialebene} an die Fläche $z=f(x,y)$ im Punkt$(x_0,y_0,z_0)$, $z_0=f(x_0,y_0)$ lautet:
\[
\boxed{\begin{aligned}	
		z&=f(x_0,y_0) + f_{'x}(x_0,y_0)(x-x_0)+ f_{'y}(x_0,y_0)(y-y_0)
	\end{aligned}}\]
\newline
\textbf{Linearisierte Gleichung} entspricht der Tangentialebene mit eingesetzten $x_0$ und $y_0$ Werten.\\
\\
Der Funktionswert ändert sich linearisiert um das \textbf{totale Differential} von $f(x,y)$
\[\boxed{
	\di z = \pdifrac{f}{x}\di x + \pdifrac{f}{y}\di y = f_{'x}(x_0,y_0)\cdot\di x + f_{'y}(x_0,y_0)\cdot\di y
}\]\\
Es wird $\di x= \Delta x$ und $\di y= \Delta y$ gesetzt.
	
\section{Implizit Ableiten}
Neue Methode um Funktionen einfacher implizit Ableiten.
Durch $F(x,y)=0$ werde implizit eine Funktion $y=f(x)$ definiert. Dann gilt für die Ableitung von $f$ an der Stelle $(x,y)$ des Graphen:
\[
\boxed{\begin{aligned}	
		y'&=-\frac{F_{'x}(x,y)}{F_{'x}(x,y)}
	\end{aligned}}\]
	
	
\section{Kettenregel}
Mit Hilfe der Kettenregel kann eine verschachtelte Funktion wie $z(t)=f(x(t),y(t))$ nach t abgeleitet werden.
\[\boxed{\begin{aligned}	
		\frac{\di z}{\di t}&= f_{'x}(x(t),y(t))\cdot \frac{\di x}{\di t}+f_{'y}(x(t),y(t))\cdot \frac{\di y}{\di t}
\end{aligned}}\]
	
\section{Extremalwert ohne Nebenbedingungen}
Der Punkt $(x_0,y_0)$ ist eine Extremstelle von $z=f(x,y)$, falls:
\[\boxed{
	f_{'x}(x_0,y_0) = 0 \qquad \text{und} \qquad f_{'y}(x_0,y_0)=0
}\] und 
zusätzlich:
\[
\boxed{\begin{array}{l}
	\Delta=f_{'xx}(x_0,y_0)\cdot f_{'yy}(x_0,y_0)-f_{'xy}(x_0,y_0)^2\\
	\Delta > 0 \text{ und }f_{'xx} < 0 \ \rightarrow\text{ Maximum}\\
	\Delta > 0 \text{ und }f_{'xx} > 0 \ \rightarrow\text{ Minimum}
	\\\\
	\Delta < 0 \ \rightarrow \text{ Sattelpunkt}
	\end{array}}\]
\\
Falls $\Delta =0$ ist, so kann man nicht entscheiden.


\section{Extremalwert mit Nebenbedingungen}
Gesucht ist die Extremstelle von $z=f(x,y)$ unter der Nebenbedingung $\varphi(x,y)=0$.\\
\[
\boxed{\begin{array}{l}
	\underline{Lagrange:}\\
	\\
	\text{Ist ($x_0$,$y_0$) eine Extremstelle, so erfüllt ($x_0$,$y_0$) das Gleichungssystem:}\\
		\begin{vmatrix}
			L_{'x}(x,y,\lambda)=0\\
			L_{'y}(x,y,\lambda)=0\\
			L_{'\lambda}(x,y,\lambda)=0\\
		\end{vmatrix}\\
		\\
		\text{mit:}\\
		L(x,y,\lambda)=f(x,y)+\lambda\cdot\varphi(x,y)\\
		f(x,y) = \text{Zielfunktion}\\
		\varphi(x,y) = \text{Nebenbedignung}\\
\end{array}}\]
\\
Die Lagrange Methode kann für mehrere Variablen und beliebig vielen Nebenbedignungen angepasst werden.\\
\\
Zielfunktion: $f(x_1,x_2,x_3,..)$\\
\\
Nebenbedingungen: 
\[
	\varphi_1(x_1,y_1)=0, \\  \varphi_2(x_2,y_2)=0
\]
\[
	L(x_1,x_2,x_3,...,\lambda,\mu,..)=f(x_1,x_2,x_3)+\lambda\cdot\varphi_1(x_1,x_2,x_3)+\mu\cdot\varphi_2(x_1,x_2,x_3)
\]
$L_{'x1}=0$\\	$L_{'x2}=0$\\
$L_{'x3}=0$ \\     $L_{'\lambda}=0$\\
$L_{'\mu}= 0$
