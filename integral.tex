% coding:utf-8

%----------------------------------------
%FOSAPHY, a LaTeX-Code for a summary of mathematics
%Copyright (C) 2013, Mario Felder, Michael Fallegger

%This program is free software; you can redistribute it and/or
%modify it under the terms of the GNU General Public License
%as published by the Free Software Foundation; either version 2
%of the License, or (at your option) any later version.

%This program is distributed in the hope that it will be useful,
%but WITHOUT ANY WARRANTY; without even the implied warranty of
%MERCHANTABILITY or FITNESS FOR A PARTICULAR PURPOSE.  See the
%GNU General Public License for more details.
%----------------------------------------

\chapter{Integrale}
\section{Allgemeine Flächenintegrale}
Schwerpunkt einer Fläche:
\[s_x=\frac{\iint x \di A}{A}\]
\[s_y=\frac{\iint y \di A}{A}\]

	
\section{Doppelintegrale}
\[ I= \int\limits_{x=a}^{b} Q(x) \di x=\int\limits_{x=a}^{b} \left( \int\limits_{y=f_{u(x)}}^{f_{o(x)}}f(x,y)\di y\right) \di x\]
\\
Integrationsreihenfolge vertauschen:
\[ \iint f(x,y)\di A=\int\limits_{y=\alpha}^{\beta} \left( \int\limits_{x=g_{l(y)}}^{g_{r(y)}}f(x,y)\di x\right) \di y\]
\\
Integrieren mit Polarkoordinaten:
\[x=r\cdot\cos \varphi\]
\[y=r\cdot\sin \varphi\]
\[ \int\limits_{\varphi=\varphi_1}^{\varphi_2} \left( \int\limits_{r=r_{u(\varphi)}}^{r_{o(\varphi)}}f(r\cdot\cos\varphi,r\cdot\sin\varphi )r\cdot\di r\right) \di \varphi\]
\\
Volumenintegrale: