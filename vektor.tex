% coding:utf-8

%----------------------------------------
%FOSAPHY, a LaTeX-Code for a summary of mathematics
%Copyright (C) 2013, Mario Felder, Michael Fallegger

%This program is free software; you can redistribute it and/or
%modify it under the terms of the GNU General Public License
%as published by the Free Software Foundation; either version 2
%of the License, or (at your option) any later version.

%This program is distributed in the hope that it will be useful,
%but WITHOUT ANY WARRANTY; without even the implied warranty of
%MERCHANTABILITY or FITNESS FOR A PARTICULAR PURPOSE.  See the
%GNU General Public License for more details.
%----------------------------------------

\chapter{Vektorgeometrie}

\section{Parameterform}
$\overrightarrow{r(t)}=
\begin{pmatrix}
 	x_{Start}\\
	y_{Start}\\
	z_{Start}
\end{pmatrix}
+
t\cdot
\begin{pmatrix}
 	x_{Ziel}-x_{Start}\\
	y_{Ziel}-y_{Start}\\
	z_{Ziel}-z_{Start}
\end{pmatrix}
$
\\
$t\in[0,1]$

\section{Geschwindigkeit, Beschleunigung}
Geschwindigkeit (Tangentialvektor):
\\
Es sei $\overrightarrow{r(t)}=
\begin{pmatrix}
 	x(t)\\
	y(t)\\
	z(t)
\end{pmatrix}$

eine Bewegung. Dann beträgt die Geschwindigkeit:
 $\overrightarrow{v(t)}=\dot{\overrightarrow{r(t)}}
\begin{pmatrix}
 	\dot{x}(t)\\
	\dot{y}(t)\\
	\dot{z}(t)
\end{pmatrix}$
und die Beschleunigung:
 $\overrightarrow{a(t)}=\dot{\overrightarrow{v(t)}}=\ddot{\overrightarrow{r(t)}}
\begin{pmatrix}
 	\dot{x}(t)\\
	\dot{y}(t)\\
	\dot{z}(t)
\end{pmatrix}$
\\
\section{Bogenlänge}
\[
	s= \int\limits_{t_1}^{t_2} v(t)\di t= \int\limits_{t_1}^{t_2} \sqrt{\dot{x}^2+\dot{y}^2+\dot{z}^2}\di t
\]


\section{Wegintegrale, Kurvenintegrale}
Um das Wegintegral zu berechnen verwendet man das newtonische Gesetz; $W=F\cdot s$
\[
	W=\int \overrightarrow{F}(\overrightarrow{r}) \di \overrightarrow{r}
\]
Anders ausgedrückt lautet die Formel:
\[
	W=\int \overrightarrow{F}(\overrightarrow{r})\cdot \dot{\overrightarrow{r}} \di t
\]
Für die Arbeit: 
\[
	A= \int\limits_{P_1}^{P_2}\overrightarrow{v}\di r= \int\limits_{t_1}^{t_2} \overrightarrow{v}\dot{\overrightarrow{r}}\di t
\]
\section{Konservative Felder, Gradientenfelder, Potentialfelder}
Ein Feld 
 $\overrightarrow{F}
\begin{pmatrix}
 	F_{x(x,y,z)}\\
	F_{y(x,y,z)}\\
	F_{z(x,y,z)}
\end{pmatrix}$ heisst konservatives Feld, wenn es eine Funktion $\varphi(x,y,z)$ so gibt dass
$	\begin{pmatrix}
	 	\varphi_{'x}\\
		\varphi_{'y}\\
		\varphi_{'z}
	\end{pmatrix} 
	=																\begin{pmatrix}
		F_{x(x,y,z)}\\											F_{y(x,y,z)}\\
		F_{z(x,y,z)}								
	\end{pmatrix} 
$	\\
\[
	\\ Grad \varphi = \overrightarrow{F}(\overrightarrow{r})
\]
\\
$\varphi$ heisst Potential.\\
In einem Potentialfeld (ohne Loch) sind die Wegintegrale wegunabhängig und es gilt:
\[
	W=\int \overrightarrow{F}(\overrightarrow{r}) \di \overrightarrow{r}= \varphi(B)-\varphi(A)
\]
Merke: Geschlossener Weg in einem Potentialfeld = 0:
\[
	W=\int \overrightarrow{F}(\overrightarrow{r}) \di \overrightarrow{r}= \oint  \overrightarrow{F}(\overrightarrow{r}) \di \overrightarrow{r}=0
\]
\\
\subsection{Konservativ}
Nach Satz von Schwarz muss gelten:\\
\[
\left.\begin{array}{l}
	\varphi_{'xy}=	\varphi_{'yx} \rightarrow \frac{\delta  F_x}{\delta y}=\frac{\delta  F_y}{\delta x} 
	\\ \\
	\varphi_{'xz}=	\varphi_{'zx} \rightarrow \frac{\delta  F_x}{\delta z}=\frac{\delta  F_z}{\delta x}
	\\ \\
	\varphi_{'yz}=	\varphi_{'zy} \rightarrow \frac{\delta  F_y}{\delta z}=\frac{\delta  F_z}{\delta y}
\end{array} \right\}
\begin{array}{l}
	\text{Integrabilitätsbedingung,}\\
	\text{falls erfüllt existiert ein Potential.}
\end{array}
\]
\\
\\
ist F konservativ.\\
\\
\\
Um das \underline{$\varphi$ zu berechnen} geht man von folgender Bedingung aus:\\
$	\begin{pmatrix}
	 	\varphi_{'x}\\
		\varphi_{'y}\\
		\varphi_{'z}
	\end{pmatrix} 
	=														\begin{pmatrix}
		F_{x(x,y,z)}\\										F_{y(x,y,z)}\\
		F_{z(x,y,z)}								
	\end{pmatrix} 
$
Somit lautet das Lösungssystem:
\[	
		\varphi_{'x} \\=\\ 	F_{x(x,y,z)} \\ \rightarrow \\ \varphi_{x} \\=\\ 	\int F_{x(x,y,z)}\di x +h(y,z)
\]
\[	
		\varphi_{'y} \\=\\  F_{y(x,y,z)} \\ \rightarrow \\
		\varphi_{y} \\=\\  \int F_{y(x,y,z)} \di y +h(x,z)
\]
\[
		\varphi_{'z} \\=\\  F_{z(x,y,z)} \\ \rightarrow \\
		\varphi_{z} \\=\\  \int F_{z(x,y,z)} \di z +h(x,y)
\]
Aus diesen 3 Gleichungen setzt man dann $\varphi(x,y,z)$ zusammen. Überschneiden sich Funktion, werden sie nicht hinzugefügt. zB $\varphi_{x}=x^2y$ und $\varphi_{y}=x^2y +2y$ ergeben: $\varphi_{x,y}=x^2y +2y+C$.



\subsection{Gradientenfelder}
Richtungsableitungen:
\\
\[
	\overrightarrow{t}=\dot{\overrightarrow{r}}=
	\begin{pmatrix}
	 	\dot{x}(t)\\
		\dot{y}(t)\\
		\dot{z}(t)
	\end{pmatrix}
	=
	\frac{\di z}{\di t}=f_{'x}\dot x +f_{'y} \dot y
	=
	Gradf\cdot \frac{\dot{\overrightarrow{r}}}{\left| \dot{\overrightarrow{r}}\right| }
	=
	\underbrace{
		\begin{pmatrix}
				 	f_{'x}\\
					f_{'y}
			\end{pmatrix}
	}_{Gradf}
	\circ
	\begin{pmatrix}
			 	\dot x\\
				\dot y
		\end{pmatrix}	
\]
\\
\[
	\overrightarrow{t}=	
	\begin{pmatrix}
		a_x\\
		a_y\\
		grad f\cdot \overrightarrow{a}
	\end{pmatrix}
\]
\\
Die Ableitung von $z=f(x,y)$ in Richtung $	\overrightarrow{a}$ beträgt $\frac{\delta f}{\delta \overrightarrow{a}}= f\cdot \overrightarrow{a} = gradf\cdot\overrightarrow{e_a}=\tan\alpha$.
\\
\\
Daraus lässt sich schliessen, dass somit 
\[
	\overrightarrow{e_a}=	
	\begin{pmatrix}
		\cos\varphi\\
		\sin\varphi
	\end{pmatrix}
\]
\\
Der Gradient steht im jeweiligen Punkt senkrecht zur Höhenlinie durch diesen Punkt. Er zeigt in Richtung grössten Anstieg. Die Falllinie zeigt in Richtung $-grad f$.
\\
Verschiedene Zusammenhänge mit gradf:
\[
	gradf \cdot gradf = f_{'x}^2+f_{'y}^2
\]
\[
	gradf\cdot \frac{	\overrightarrow{gradf}}{|gradf|}=|gradf|
\]
\[
	\overrightarrow{a} \cdot	\overrightarrow{b} = 	|\overrightarrow{a}| \cdot	|\overrightarrow{b}|\cdot \cos \varphi
\]