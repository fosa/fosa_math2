% coding:utf-8

%----------------------------------------
%FOSAPHY, a LaTeX-Code for a summary of mathematics
%Copyright (C) 2013, Mario Felder, Michael Fallegger

%This program is free software; you can redistribute it and/or
%modify it under the terms of the GNU General Public License
%as published by the Free Software Foundation; either version 2
%of the License, or (at your option) any later version.

%This program is distributed in the hope that it will be useful,
%but WITHOUT ANY WARRANTY; without even the implied warranty of
%MERCHANTABILITY or FITNESS FOR A PARTICULAR PURPOSE.  See the
%GNU General Public License for more details.
%----------------------------------------

\chapter{Integrale}

\section{Doppelintegrale}
Kartesische Koordinaten
\[\boxed{
	\iint\limits_A f(x,y) \di A = \underbrace{ \int\limits_{x=a}^{b}\ \underbrace{ \int\limits_{y=f_{u}(x)}^{f_{o}(x)} f(x,y)\ \di y}_{inneres Integral} \ \di x}_{äusseres Integral}
}\]
\\
Integrationsreihenfolge vertauschen:
\[\boxed{
	\iint\limits_A f(x,y)\di A=\int\limits_{y=a}^{b}\  \int\limits_{x=g_{l}(y)}^{g_{r}(y)}f(x,y)\ \di x \di y
}\]
\\
\begin{samepage}
	Polarkoordinaten:
	\[
		x=r\cdot\cos \varphi	\\	y=r\cdot\sin \varphi
	\]
	\[\boxed{ 
		\int\limits_{\varphi=\varphi_1}^{\varphi_2} \left( \int\limits_{r=r_{i}(\varphi)}^{r_{a}(\varphi)}f(r\cdot\cos\varphi,r\cdot\sin\varphi )\cdot r \cdot\di r\right) \di \varphi
	}\]
\end{samepage}
\\

\section{Allgemeine Flächenintegrale}
\begin{tabular}{p{.5\linewidth}p{.5\linewidth}}
	Kartesische Koordinaten:	&	Polarkoordinaten:\\
	\[\boxed{
		A = \iint\limits_{A} 1\ \di A
	}\]
	&
	\[\boxed{
		A = \iint\limits_{A} r\ \di A
	}\]
\end{tabular}


\subsection{Schwerpunkt einer Fläche}
Kartesische Koordinaten:
\[\boxed{
	s_x=\frac{1}{A}\iint\limits_{A} x \di A \qquad
	s_y=\frac{1}{A}\iint\limits_{A} y \di A
}\]
\\
Polarkoordinaten:
\[\boxed{
	s_x=\frac{1}{A}\iint\limits_{A} r^2 \cdot \cos(\varphi) \di A \qquad
	s_y=\frac{1}{A}\iint\limits_{A} r^2 \cdot \sin(\varphi) \di A
}\]

\subsection{Flächenträgheitsmoment}
\begin{tabular}{p{.5\linewidth} p{.5\linewidth}}
	Kartesische Koordinaten:	&	Polarkoordinaten: \\
	\[\boxed{\begin{aligned}
		I_x &= \iint\limits_A y^2 \di A\\
		I_y &= \iint\limits_A x^2 \di A\\
		I_{p}&=\iint\limits_A (x^2+y^2) \di A
	\end{aligned}}\]	
	&	
	\[\boxed{\begin{aligned}
		I_x &= \iint\limits_A r^3 \cdot \sin^2(\varphi) \di A\\
		I_y &= \iint\limits_A r^3 \cdot \cos^2(\varphi) \di A\\
		I_{p}&=\iint\limits_A r^3 \di A
	\end{aligned}}\]
\end{tabular}
\\

\section{Volumenintegrale}
Beim Volumenintegral wird über die Projektionsfläche A integriert.\\
\\
Kartesische Koordinaten:
\[\boxed{
	\iiint\limits_V f(x,y,z) \di V = 
	\int\limits_{x=a}^{b}\ 
	\int\limits_{y=f_u(x)}^{f_o(x)}\  \int\limits_{z=z_{u(x,y)}}^{z_{o(x,y)}}
	f(x,y,z)\ \di z\ \di y\ \di x\
}\]
\\
Bei Rotationskörper gilt für die Grenzen von $z$:\\
\[
	z = f(x) = f(\sqrt{x^2 + y^2})
\]
\\
Zylinderkoordinaten:
\[\boxed{
	\iiint\limits_V f(x,y,z) \di V = 
	\iiint\limits_V	f(r \cdot \cos\varphi,r \cdot \sin\varphi,z) \cdot r\ \di z\ \di r\ \di \varphi\
}\]
\\
Kugelkoordinaten:
\[\boxed{
	\iiint\limits_V f(x,y,z) \di V = 
	\iiint\limits_V	
	f\left(\begin{matrix}
	r \cdot \sin\vartheta \cdot \cos\varphi\\
	r \cdot \sin\vartheta \cdot \sin\varphi\\
	r \cdot \cos\vartheta
	\end{matrix}\right) \cdot r^2\sin\vartheta\ \di r\ \di \vartheta\ \di \varphi\
}\]
\\

\section{Allgemeine Volumenintegrale}
\begin{tabular}{p{.5\linewidth}p{.5\linewidth}}
	Kartesische Koordinaten:	&	Rotationskörper:\\
	\[\boxed{
		V = \iiint\limits_{V} 1\ \di V
	}\]
	&
	\[\boxed{
		V = \iiint\limits_{V} r\ \di V
	}\]
\end{tabular}


\subsection{Schwerpunkt eines homogenen Körpers}
Kartesische Koordinaten:
\[\boxed{
	s_x=\frac{1}{V}\iiint\limits_{V} x \ \di V \qquad
	s_y=\frac{1}{V}\iiint\limits_{V} y \ \di V \qquad
	s_z=\frac{1}{V}\iiint\limits_{V} z \ \di V
}\]
\\
Rotationskörper:
\[\boxed{
	s_x = 0 \qquad
	s_y = 0 \qquad
	s_z=\frac{1}{V}\iiint\limits_{V} zr \ \di V
}\]

\subsection{Massenträgheitsmoment}
\begin{tabular}{p{.5\linewidth}p{.5\linewidth}}
	Kartesische Koordinaten:	&	Rotationskörper: \\
	\[\boxed{
		I = \rho \cdot \iiint\limits_V (x^2+y^2) \di V
	}\]	
	&	
	\[\boxed{
		I = \rho \cdot \iiint\limits_V r^3 \ \di V
	}\]
\end{tabular}