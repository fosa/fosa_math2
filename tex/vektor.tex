% coding:utf-8

%----------------------------------------
%FOSAPHY, a LaTeX-Code for a summary of mathematics
%Copyright (C) 2013, Mario Felder, Michael Fallegger

%This program is free software; you can redistribute it and/or
%modify it under the terms of the GNU General Public License
%as published by the Free Software Foundation; either version 2
%of the License, or (at your option) any later version.

%This program is distributed in the hope that it will be useful,
%but WITHOUT ANY WARRANTY; without even the implied warranty of
%MERCHANTABILITY or FITNESS FOR A PARTICULAR PURPOSE.  See the
%GNU General Public License for more details.
%----------------------------------------

\chapter{Vektorgeometrie}

\section{Parameterform}
\[\overrightarrow{r(t)}=
\begin{pmatrix}
 	x_{Start}\\
	y_{Start}\\
	z_{Start}
\end{pmatrix}
+
t\cdot
\begin{pmatrix}
 	x_{Ziel}-x_{Start}\\
	y_{Ziel}-y_{Start}\\
	z_{Ziel}-z_{Start}
\end{pmatrix}
\]
\\
$t\in[0,1]$

\section{Geschwindigkeit, Beschleunigung}
Geschwindigkeit (Tangentialvektor):
\\
Es sei $\overrightarrow{r(t)}=
\begin{pmatrix}
 	x(t)\\
	y(t)\\
	z(t)
\end{pmatrix}$ eine Bewegung. Dann beträgt die Geschwindigkeit:
 $\overrightarrow{v(t)}=\dot{\overrightarrow{r(t)}}
\begin{pmatrix}
 	\dot{x}(t)\\
	\dot{y}(t)\\
	\dot{z}(t)
\end{pmatrix}$
und die Beschleunigung:
 $\overrightarrow{a(t)}=\dot{\overrightarrow{v(t)}}=\ddot{\overrightarrow{r(t)}}
\begin{pmatrix}
 	\ddot{x}(t)\\
	\ddot{y}(t)\\
	\ddot{z}(t)
\end{pmatrix}$
\\
\section{Bogenlänge}
\[\boxed{
	s= \int\limits_{t_1}^{t_2} v(t)\di t= \int\limits_{t_1}^{t_2} \sqrt{\dot{x}^2+\dot{y}^2+\dot{z}^2}\di t
}\]

\section{Vektorfeld}
Ebenes Vektorfeld:
\[\boxed{
	\vec{F} = F_x(x,y) \cdot \vec{e}_x + F_y(x,y) \cdot \vec{e}_y = \left( \begin{matrix}
		F_x(x,y)\\
		F_y(x,y)
	\end{matrix} \right)
}\]
\\
\begin{footnotesize}
	$F_x,F_y$: Skalare Komponenten des Vektorfeldes $\vec{F}(x,y)$
\end{footnotesize}
\\

\section{Wegintegrale, Kurvenintegrale}
Um das Wegintegral zu berechnen verwendet man das newtonische Gesetz: $W=F\cdot s$
\[\boxed{
	W=\int\limits_C \overrightarrow{F} \ \di \overrightarrow{r} = 
	\int\limits_{t_1}^{t_2} \overrightarrow{F}\cdot \dot{\overrightarrow{r}} \ \di t =
	\int\limits_{t_1}^{t_2} \left( F_x \cdot \dot{x} + F_y \cdot \dot{y} \right) \ \di t
}\]
\\
\begin{footnotesize}
	$\vec{F} = \vec{F}(x,y)$: ebenes Fektorfeld\\
	$\vec{r} = \vec{r}(t)$: Ortsvektor der Kurve $C$\\ 
\end{footnotesize}
\\
Für die Arbeit: 
\[
	A= \int\limits_{P_1}^{P_2}\overrightarrow{v}\di r= \int\limits_{t_1}^{t_2} \overrightarrow{v}\dot{\overrightarrow{r}}\di t
\]
\\


\section{Gradient eines Skalarfeldes}
\[\boxed{
	\renewcommand{\arraystretch}{1.4}
	\grad \varphi = \pdifrac{\varphi}{x} \cdot \vec{e}_x + \pdifrac{\varphi}{y} \cdot \vec{e}_y + \pdifrac{\varphi}{z} \cdot \vec{e}_z = 
	\left(\begin{matrix}
		\pdifrac{\varphi}{x} \\
		\pdifrac{\varphi}{y} \\
		\pdifrac{\varphi}{z}
	\end{matrix}\right)
}\]
\\
\begin{footnotesize}
	$\varphi = \varphi(x,y,z)$: Räumliches Skalarfeld
\end{footnotesize}


\section{Konservative Felder, Potentialfelder}
Ein Vektorfeld $\vec{F}$ heisst konservatives Feld oder Potentialfeld, wenn es eine Funktion $\varphi(x,y,z)$ so gibt dass:
\[\boxed{
	\renewcommand{\arraystretch}{1.4}
	\begin{aligned}	
	\vec{F} &= \grad \varphi \\
	\left( \begin{matrix}
		F_x \\
		F_y \\
		F_z \\
	\end{matrix} \right) &=
	\left(\begin{matrix}
		\pdifrac{\varphi}{x} \\
		\pdifrac{\varphi}{y} \\
		\pdifrac{\varphi}{z}
	\end{matrix}\right)
\end{aligned}}\]
\\
\begin{footnotesize}
	$\varphi$: heisst Potential.\\
\end{footnotesize}
\\
In einem Potentialfeld (ohne Loch) sind die Wegintegrale wegunabhängig und es gilt:
\[
	W=\int \overrightarrow{F} \ \di \overrightarrow{r}= \phi(B)-\phi(A)
\]
Merke: Geschlossener Weg in einem Potentialfeld = 0:
\[
	W=\int \overrightarrow{F} \ \di \overrightarrow{r}= \oint  \overrightarrow{F} \ \di \overrightarrow{r}=0
\]
\\
\subsection{Konservativ}
Nach Satz von Schwarz muss gelten:\\
\[
\left.\begin{array}{l}
	\varphi_{'xy}=	\varphi_{'yx} \rightarrow \pdifrac{F_x}{y} = \pdifrac{F_y}{x} 
	\\ \\
	\varphi_{'xz}=	\varphi_{'zx} \rightarrow \pdifrac{F_x}{z} = \pdifrac{F_z}{x}
	\\ \\
	\varphi_{'yz}=	\varphi_{'zy} \rightarrow \pdifrac{F_y}{z} = \pdifrac{F_z}{y}
\end{array} \right\}
\begin{array}{l}
	\text{Integrabilitätsbedingung,}\\
	\text{falls erfüllt existiert ein Potential}\\
	\text{und $\vec{F}$ ist konservativ.}
\end{array}
\]
\\
\\
Um \underline{$\varphi$ zu berechnen} geht man von folgender Bedingung aus:
\[
	\begin{pmatrix}
	 	\varphi_{'x}\\
		\varphi_{'y}\\
		\varphi_{'z}
	\end{pmatrix} 
	=														\begin{pmatrix}
		F_{x(x,y,z)}\\										F_{y(x,y,z)}\\
		F_{z(x,y,z)}								
	\end{pmatrix} 
\]
\\
Somit lautet das Lösungssystem:
\[	
		\varphi_{'x} \\=\\ 	F_{x(x,y,z)} \\ \rightarrow \\ \varphi_{x} \\=\\ 	\int F_{x(x,y,z)}\di x +h(y,z)
\]
\[	
		\varphi_{'y} \\=\\  F_{y(x,y,z)} \\ \rightarrow \\
		\varphi_{y} \\=\\  \int F_{y(x,y,z)} \di y +h(x,z)
\]
\[
		\varphi_{'z} \\=\\  F_{z(x,y,z)} \\ \rightarrow \\
		\varphi_{z} \\=\\  \int F_{z(x,y,z)} \di z +h(x,y)
\]
Aus diesen 3 Gleichungen setzt man $\varphi(x,y,z)$ zusammen. Überschneiden sich Funktion, werden sie nicht hinzugefügt.\\
\\
\begin{footnotesize}
z.B.: $\varphi_{x}=x^2y$ und $\varphi_{y}=x^2y +2y$ ergeben: $\varphi(x,y)=x^2y +2y+C$.
\end{footnotesize}
\\


\subsection{Richtungsableitung}
Richtungsableitung:
\[\boxed{
	\pdifrac{f}{\vec{a}} = \grad f \cdot \vec{e}_a = \frac{\grad f \cdot \vec{a}}{\left| \vec{a} \right|} = \tan\alpha
}\]

\[
	\vec{t}=\dot{\vec{r}}=
	\begin{pmatrix}
	 	\dot{x}(t)\\
		\dot{y}(t)\\
		\dot{z}(t)
	\end{pmatrix}
	=
	\frac{\di z}{\di t}=f_{'x}\dot x +f_{'y} \dot y
	=
	\grad f\cdot \frac{\dot{\vec{r}}}{\left| \dot{\vec{r}}\right| }
	=
	\underbrace{
		\begin{pmatrix}
				 	f_{'x}\\
					f_{'y}
			\end{pmatrix}
	}_{\grad f}
	\cdot
	\begin{pmatrix}
			 	\dot x\\
				\dot y
		\end{pmatrix}	
\]
\\
\[
	\vec{t}=	
	\begin{pmatrix}
		a_x\\
		a_y\\
		\grad f\cdot \vec{a}
	\end{pmatrix}
\]
\\
\\
Wenn der Winkel zur $x$-Achse gegeben ist für die Richtung:
\[
	\vec{e_a}=	
	\begin{pmatrix}
		\cos\alpha\\
		\sin\alpha
	\end{pmatrix}
\]
\\
Der Gradient steht im jeweiligen Punkt senkrecht zur Höhenlinie. Er zeigt in Richtung grössten Anstieg.\\
Die Falllinie zeigt in Richtung $-\grad f$.
\\
\\
Verschiedene Zusammenhänge mit $\grad f$:
\[
	\grad f \cdot \grad f = f_{'x}^2+f_{'y}^2
\]
\[
	\grad f\cdot \frac{	\vec{\grad f}}{|\grad f|}=|\grad f|
\]
\[
	\vec{a} \cdot	\vec{b} = 	\left| \vec{a} \right| \cdot	\left| \vec{b} \right| \cdot \cos \varphi
\]
